\NeedsTeXFormat{LaTeX2e}
%\PassOptionsToClass{handout}{beamer}
\documentclass{beamer}
\usepackage{beamerPack}
%\usepackage{bm}
\usepackage[02]{../lecture}
\subtitle{制御構造}
\begin{document}

\begin{frame}[fragile]{}
\titlepage
\end{frame}

\section{制御構造}		%%%%%%%%
\subsection{}

\begin{frame}[fragile]{プログラムの流れの制御}{}
\begin{itemize}\itemsep8pt
\item 繰り返す
\item 条件次第で実行する
\item メッセージを送る・受け取る
\item 名前をつける
\end{itemize}
\end{frame}

\section{繰り返し}		%%%%%%%%
\subsection{}

\begin{frame}[fragile]{}{}
\begin{exampleblock}{繰り返し}
\begin{enumerate}%\itemsep8pt
\item 回数指定(丸)、終了条件なし、終了条件あり(菱形)
\item 繰り返す「指示」
\end{enumerate}
\end{exampleblock}

\begin{center}
\includegraphics[width=0.7\pagewidth]{loop1.png}
\includegraphics[width=0.7\pagewidth]{loop2.png}
\end{center}
\end{frame}

\begin{frame}[fragile]{丸いもの}{}
\begin{itemize}%\itemsep8pt
\item 数
\item 演算(計算):緑
\item 変数
\end{itemize}
緑以外にも丸いものは色々あるので注意
\end{frame}

\begin{frame}[fragile]{菱形}{}
\begin{itemize}%\itemsep8pt
\item 大きいかどうか、小さいかどうか、等しいかどうか
\item かつ、または、ではない
\item 含まれるかどうか
\end{itemize}
どれも2択、緑以外にも菱形は色々あるので注意
\end{frame}

\begin{frame}[fragile]{繰り返しの練習}{}
\begin{itemize}%\itemsep8pt
\item
どこかに行くを無限に繰り返しながら、各頂点で360度回るアニメーションをする
(移動の繰り返しと回転の繰り返しの関係)
\end{itemize}
\end{frame}

\section{条件実行}		%%%%%%%%
\subsection{}

\begin{frame}[fragile]{}{}
\begin{exampleblock}{条件次第で実行する:オレンジ色}
\begin{enumerate}\itemsep8pt
\item 条件(無条件も含む): 菱形
\item 条件が成立したときの「指示」
\item (条件が成立しなかったときの「指示」)
\end{enumerate}
\end{exampleblock}
\begin{center}
\includegraphics[width=0.7\pagewidth]{cond.png}
\end{center}

\end{frame}

\begin{frame}[fragile]{条件判断の練習}{}
\begin{exampleblock}{問題1}
\begin{itemize}%\itemsep8pt
\item
どこかに行くを無限に繰り返しながら、各頂点に来たときで、以下を実行
\begin{itemize}%\itemsep8pt
\item 今いるところのX座標が正の数ならニャーと鳴く。
\item 今いるところのX座標が正の数でないなら何もしない。
\end{itemize}
\end{itemize}
\end{exampleblock}

\begin{exampleblock}{問題2}
\begin{itemize}%\itemsep8pt
\item
どこかに行くを無限に繰り返しながら、各頂点に来たとき、以下を実行
\begin{itemize}%\itemsep8pt
\item 今いるところのX座標が正の数ならニャーと鳴く。
\item 今いるところのX座標が正の数でないならペンの色を少し変える
\end{itemize}
\end{itemize}
\end{exampleblock}
\end{frame}

\begin{frame}[fragile]{様々な条件}{}
\begin{itemize}%\itemsep8pt
\item マウスのX座標やY座標、距離などを使った計算結果:マウスの場所まで近いかどうか
\item 別のスプライトのX座標やY座標、距離などを使った計算結果:別のスプライトに衝突したかどうか
\item キーボードで入力した文字の判定:スペースキーが今推されているかどうか
\end{itemize}
\end{frame}

\begin{frame}[fragile]{様々な条件}{}

\begin{exampleblock}{マウスを使った条件}
\begin{itemize}%\itemsep8pt
\item
どこかに行くを無限に繰り返しながら、各頂点に来たとき、以下の2択を実行
\begin{itemize}%\itemsep8pt
\item マウスとの距離が20以下ならニャーと鳴く。
\item そうでないなら何もしない。
\end{itemize}
\end{itemize}
\end{exampleblock}

\begin{exampleblock}{別のスプライトとの衝突}
スプライトを2つ追加します。
\begin{itemize}%\itemsep8pt
\item
どこかに行くを無限に繰り返しながら、各頂点に来たとき、以下の2択を実行
\begin{itemize}%\itemsep8pt
\item 追加したスプライトとの距離が50以下ならニャーとなく。
\item そうでないなら何もしない。
\end{itemize}
\end{itemize}
\end{exampleblock}
\end{frame}

\begin{frame}[fragile]{}{}
もっといろいろな条件を使って動作を変更したい
\begin{itemize}%\itemsep8pt
\item 辺の数が0や1なら、描けないので、表示しない
\item 辺の数が20以上なら、区別がつかないので、表示しない
\item 奇数番目の辺と偶数番目の辺で、ペンの色を変えたい(2択)
\end{itemize}
そのようなブロックはない
\end{frame}

\begin{frame}[fragile]{変数}{}
数値を覚えるための「変数」を作ることができる
\begin{itemize}%\itemsep8pt
\item 「辺の数」用の変数を作る
\item 「何番目の辺か」用の変数を作る
\end{itemize}

\begin{center}
\includegraphics[width=0.7\pagewidth]{var.png}
\end{center}

\vfill
変数を作ったら、条件で使える
\end{frame}

\begin{frame}[fragile]{}{}
偶数かそうでないかを条件として使う場合
\begin{enumerate}\itemsep8pt
\item 何番目の辺かを表す変数を作る
\item 最初に何番目の辺かを表す変数を0に設定する
\item 辺を描いたら何番目の辺かを表す変数を1増やす
\item 何番目の辺かを表す変数が偶数か奇数かで2択をする(2で割り切れたら偶数)
\end{enumerate}
\end{frame}

\begin{frame}[fragile]{演習}{}
\begin{exampleblock}{特別扱い}
どこかに行くを無限に繰り返しながら、3個目の頂点に来たとき「ニャー」となく
\end{exampleblock}

\pause

\begin{exampleblock}{マウスを使った条件の高度化}
\begin{itemize}%\itemsep8pt
\item
どこかに行くを無限に繰り返しながら、各頂点に来たとき、以下の2択を実行
\begin{itemize}%\itemsep8pt
\item マウスとの距離が20以下で、しかも=「かつ」まだ1度も鳴いていない、ならば「ニャー」と鳴く
\item そうでないなら何もしない
\end{itemize}
\end{itemize}
\end{exampleblock}
\end{frame}

\begin{frame}[fragile]{}{}
作品制作の時間
\end{frame}
\end{document}
