\NeedsTeXFormat{LaTeX2e}
%\PassOptionsToClass{handout}{beamer}
\documentclass{beamer}
\usepackage{beamerPack}
\usepackage{bm}
\usepackage[01]{../lecture}
\subtitle{プログラミングとは並べること}
\begin{document}

\begin{frame}[fragile]{}
%\stain{0.01}{0.33}{20}{0.5\textwidth}{0.26\textheight}
\titlepage
\end{frame}

\begin{frame}[fragile]{outline}{}
\begin{tikzpicture}[
    overlay
    , xshift=0.423\pagewidth
    , yshift=-0.3\pageheight
]
\begin{pgfonlayer}{background}
\node[anchor=center] at (0,0.7) {\includegraphics[width=1.1\pagewidth]{ben-maguire-268306.jpg}};
\end{pgfonlayer}
\end{tikzpicture}
\fontsize{8}{10}\selectfont
\vfill
\textcnd{\textcolor{white}{情報を処理}}

\vfill
\textcnd{\textcolor{white}{コンピュータをプログラミング}}

\vfill
\textcolor{white}{アイデアの実現}

\vfill
\end{frame}

\section{イントロ}		%%%%%%%%
\subsection{}

\begin{frame}[fragile]{}{}
コンピュータやスマートフォンにしかできないこと

\begin{itemize}%\itemsep8pt
\item 3D CG
\item ゲーム
\item チャットなど
\end{itemize}

\vfill
\pause
プログラムがあればコンピュータにできること

\begin{itemize}%\itemsep8pt
\item 3D CG
\item ゲーム
\item チャットなど
\end{itemize}

\vfill
\pause
プログラムを作ればそれまでできなかったことができる

\begin{itemize}%\itemsep8pt
\item 自分で作ったゲーム
\item 自分で作った動画編集プログラム
\end{itemize}
\end{frame}

\begin{frame}[fragile]{プログラムを作ろう}{}

いろいろなコンピュータへの「指示」が用意されている

「指示」を並べると「プログラム」になる

難しい?
\end{frame}


\begin{frame}[fragile]{Scratch}{}
これから使うのはScratchというプログラミングのためのアプリ

\vfill
\begin{itemize}\itemsep8pt
\item 「指示」の数はそれほど多くない
\item 「指示」が日本語の説明付きのタイルになっている
\item タイルの色や形が並べるときのヒントになっている
\end{itemize}
\end{frame}

\section{実践}		%%%%%%%%
\subsection{}


\begin{frame}[fragile]{イントロ}{}
\begin{itemize}%\itemsep8pt
\item 画面の見方
\item 実行と停止
\end{itemize}
\end{frame}

\begin{frame}[fragile]{イントロ}{}
\tikz[FULLSCREEN]{\node at (0,0) {\pgfimage[width=0.9\pagewidth]{Scratch-screen.png}};}
\end{frame}

\begin{frame}[fragile]{プログラムの作成とプログラムの実行開始}{}
\begin{enumerate}\itemsep20pt
\item 旗をクリックしたら実行が始まる機能
\item 別の場所に移動する
\item 移動が終わったらアクション
\end{enumerate}

\vfill
\pause
3つの組み合わせ

\vfill
\pause
\textbyhand{人生初のプログラムは3つの「指示」を組み合わせました。}
\end{frame}

\begin{frame}[fragile]{プログラムの改良}{}
\begin{enumerate}[<+->]\itemsep20pt
\item 旗をクリックしたら実行が始まる機能
\item 別の場所に、時間を掛け、中間地点を経由しながら、移動する
\item 移動が終わったらアクション
\end{enumerate}
\vfill
\pause
プログラムの変更が必要:使う指示の数を増やしてよい
\vfill
\pause
指示が長くなった $\to$ 難しい注文に対応できた
\end{frame}

\begin{frame}[fragile]{応用}{}
\begin{itemize}[<+->]\itemsep30pt
\item
(動き方は自由でいいので)別の場所に移動して、アクションしてから、同じように移動して元の場所に戻ってくる
\item
別の場所に移動して元の場所に戻ってくるを3回繰り返す
\item
別の場所に移動して元の場所に戻ってくるを無限に繰り返す
\end{itemize}

\vfill
いろいろな「指示」の中で使えそうなものをことごとく試さなければ。
\end{frame}

\begin{frame}[fragile]{まだまだ応用}{}
\begin{enumerate}\itemsep8pt
\item 加速しながら移動しているように見せるには?
\item 正方形を描いて正確に元の場所に戻ってくるには?
\item 正三角形を描いて正確に元の場所に戻ってくるには?
\item 正$N$角形を描いて正確に元の場所に戻ってくるには?
\item 移動する前にキーボードでNの値を指示できるようにするには?
\item 別のキーを押すと強制的に元の位置に戻るようにするには?
\item キーを押している間だけ移動速度が加速するには?
\end{enumerate}
\end{frame}

\begin{frame}[fragile]{加速しながら移動しているように見せるには?}{}

そのような「指示」はないので工夫が必要。

\vfill
先ほど「ゆっくり移動」させるように改良できた。応用できないか?

\vfill
\pause
「指示」の組み合わせ方の工夫+「指示」への指示の工夫
\end{frame}


\begin{frame}[fragile]{まだまだ応用}{}
\begin{enumerate}\itemsep8pt
\item \textcolor{red}{(終了)}\textcolor{gray}{加速しながら移動しているように見せるには?}
\item 正方形を描いて正確に元の場所に戻ってくるには?
\item 正三角形を描いて正確に元の場所に戻ってくるには?
\item 正$N$角形を描いて正確に元の場所に戻ってくるには?
\item 移動する前にキーボードでNの値を指示できるようにするには?
\item 別のキーを押すと強制的に元の位置に戻るようにするには?
\item キーを押している間だけ移動速度が加速するには?
\end{enumerate}
\end{frame}


\section{まとめ}		%%%%%%%%
\subsection{}

\begin{frame}[fragile]{到達目標}{}
\begin{itemize}[<+->]\itemsep8pt
\item Scratchってなに?
\item Scratchってどうすれば使える?
\item Scratchで画面のどれが「指示」?
\item Scratchの画面のどこが自分で作ったプログラム?
\item どんなプログラムができる?ゲームはできた?
\end{itemize}
\end{frame}

\end{document}
