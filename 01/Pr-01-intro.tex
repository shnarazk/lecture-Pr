\NeedsTeXFormat{LaTeX2e}
%\PassOptionsToClass{handout}{beamer}
\documentclass{beamer}
\usepackage{beamerPack}
\usepackage{bm}
\usepackage[01]{../lecture}
\subtitle{プログラミングとは並べること}
\begin{document}

\begin{frame}[fragile]{}
\titlepage
\end{frame}

\section{イントロ}		%%%%%%%%
\subsection{}

\begin{frame}[fragile]{}{}
コンピュータやスマートフォンにしかできないこと

\begin{itemize}%\itemsep8pt
\item 3D CG
\item ゲーム
\item チャットなど
\end{itemize}

\vfill
\pause
プログラムがあればコンピュータにできること

\begin{itemize}%\itemsep8pt
\item 3D CG
\item ゲーム
\item チャットなど
\end{itemize}

\vfill
プログラムを作ればできなかったことができる

\begin{itemize}%\itemsep8pt
\item 自分で作ったゲーム
\item 自分で作った動画編集プログラム
\end{itemize}
\end{frame}

\begin{frame}[fragile]{プログラムを作ろう}{}
\begin{itemize}\itemsep20pt
\item いろいろなコンピュータへの「指示」が用意されている
\item 「指示」を並べると「プログラム」になる
\end{itemize}

\begin{block}{Scratch}
\begin{itemize}\itemsep8pt
\item 「指示」の数はそれほど多くない
\item 「指示」が日本語の説明付きのタイルになっている
\item タイルの色や形が並べるときのヒントになっている
\end{itemize}
\end{block}
\end{frame}

\section{Scratch実践}		%%%%%%%%
\subsection{}

\begin{frame}[fragile]{イントロ}{}
\tikz[FULLSCREEN]{\node at (0,0) {\pgfimage[width=0.9\pagewidth]{Scratch-screen.png}};}
\end{frame}

\begin{frame}[fragile]{プログラムの作成とプログラムの実行開始}{}
\begin{enumerate}\itemsep20pt
\item 旗をクリックしたら実行が始まる機能
\item 別の場所に移動する
\item 移動が終わったらアクション
\end{enumerate}

\vfill
3つの「指示」の組み合わせで構成されたプログラム

\end{frame}

\begin{frame}[fragile]{プログラムの拡張1}{}
\begin{enumerate}\itemsep20pt
\item 旗をクリックしたら実行が始まる機能
\item 別の場所に移動する
\item 移動が終わったらアクション
\item さらに別の場所に移動する
\end{enumerate}
\vfill
プログラムの変更が必要:使う指示の数を増やしてよい
\vfill
\pause
指示が長くなった $\to$ プログラムの高度化
\end{frame}

\begin{frame}[fragile]{プログラムの拡張2}{}
\begin{itemize}\itemsep30pt
\item
「別の場所に移動して、アクションする」を3回繰り返す
\item
「別の場所に移動して、アクションする」を無限に繰り返す
\end{itemize}

\vfill
いろいろな「指示」の中で使えそうなものを試してみる
\end{frame}

\begin{frame}[fragile]{プログラムの拡張3}{}
\begin{exampleblock}{動いた軌跡を表示しよう}
\begin{itemize}\itemsep10pt
\item
拡張機能ペン
\item
ペンを下ろして移動すると軌跡が残る
\end{itemize}
\end{exampleblock}

\begin{itemize}\itemsep20pt
\item
「軌跡を残しながら別の場所に移動して、アクションする」を3回繰り返す
\item
「軌跡を残しながら別の場所に移動して、アクションする」を無限に繰り返す
\end{itemize}
\end{frame}

\section{課題}		%%%%%%%%
\subsection{}

\begin{frame}[fragile]{全く違う応用}{}
\begin{exampleblock}{同時進行}
\begin{enumerate}\itemsep8pt
\item コスチュームを使って移動するときは歩くアニメーションをする
\item 移動し終わったらアニメーションも停止する
\end{enumerate}
\end{exampleblock}
\end{frame}

\begin{frame}[fragile]{演習課題}{}
\begin{enumerate}\itemsep8pt
\item 正方形を描いて正確に元の場所に戻ってくるには?
\begin{itemize}%\itemsep8pt
\item そのような「指示」はないので工夫が必要
\item どこかへの移動ではなく、自分で決めた1辺の長さ(座標)で移動場所を考える
\end{itemize}

\item ゆっくり正方形を描いて正確に元の場所に戻ってくる
\begin{itemize}%\itemsep8pt
\item どうやって時間を掛けて移動すればいいだろうか
\end{itemize}
\item ゆっくり正三角形を描いて正確に元の場所に戻ってくる
\item ゆっくり正5角形を描いて正確に元の場所に戻ってくる
\end{enumerate}
\end{frame}

\section{まとめ}		%%%%%%%%
\subsection{}

\begin{frame}[fragile]{確認項目}{}
\begin{itemize}\itemsep18pt
\item Scratchってなに?
\item Scratchってどうすれば使える?
\item Scratchで画面のどれが「指示」?
\item Scratchの画面のどこが自分で作ったプログラム?
\item どんなプログラムができる?
\end{itemize}
\end{frame}

\end{document}
