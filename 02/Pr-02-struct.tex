\NeedsTeXFormat{LaTeX2e}
%\PassOptionsToClass{handout}{beamer}
\documentclass{beamer}
\usepackage{beamerPack}
\usepackage{bm}
\usepackage[01]{../lecture}
\subtitle{制御構造}
\begin{document}

\begin{frame}[fragile]{}
\titlepage
\end{frame}

\section{イントロ}		%%%%%%%%
\subsection{}

\begin{frame}[fragile]{プログラムの流れの制御}{}
\begin{itemize}\itemsep8pt
\item 繰り返す
\item 条件次第で実行する
\item メッセージを送る・受け取る
\item 名前をつける
\end{itemize}
\end{frame}

\begin{frame}[fragile]{プログラムの流れの制御}{}
\begin{exampleblock}{繰り返す}
\begin{enumerate}%\itemsep8pt
\item 条件(無条件も含む): ブロックの形は?
\item 繰り返す「指示」
\end{enumerate}
\end{exampleblock}
\end{frame}

\begin{frame}[fragile]{繰り返しの練習1}{}
\begin{itemize}%\itemsep8pt
\item
三角形を書きながら、各頂点で360度回るアニメーションをする
\item 
ヒント:繰り返しの中で繰り返し
\end{itemize}
\end{frame}

\begin{frame}[fragile]{条件次第で実行する}{}
条件文または条件分岐

\begin{exampleblock}{条件次第で実行する:ブロックの色は}
\begin{enumerate}\itemsep8pt
\item 条件(無条件も含む): ブロックの形は
\item 条件が成立したときの「指示」
\item (条件が成立しなかったときの「指示」)
\end{enumerate}
\end{exampleblock}
\end{frame}

\begin{frame}[fragile]{条件判断の練習}{}
\begin{exampleblock}{問題1}
\begin{itemize}%\itemsep8pt
\item
三角形を書きながら、各頂点で、以下の2択を実行
\begin{itemize}%\itemsep8pt
\item 今いるところのX座標が正の数ならニャーと鳴く。
\item 今いるところのX座標が正の数でないなら特に何もしない。
\end{itemize}
\end{itemize}
\end{exampleblock}

\begin{exampleblock}{問題2}
\begin{itemize}%\itemsep8pt
\item
三角形を書きながら、各頂点で、以下の2択を実行
\begin{itemize}%\itemsep8pt
\item 今いるところのX座標が正の数ならニャーと鳴く。
\item 今いるところのX座標が正の数でないならペンの色を変える
\end{itemize}
\end{itemize}
\end{exampleblock}
\end{frame}

\begin{frame}[fragile]{様々な条件}{}
\begin{itemize}%\itemsep8pt
\item スプライトのX座標やY座標、それを使った計算結果
\item マウスのX座標やY座標、距離などを使った計算結果
\item 別のスプライトのX座標やY座標、距離などを使った計算結果
\item キーボードで入力した文字
\end{itemize}

\begin{exampleblock}{問題3}
\begin{itemize}%\itemsep8pt
\item
三角形を書きながら、各頂点で、以下の2択を実行
\begin{itemize}%\itemsep8pt
\item マウスとの距離が50以下ならニャーと鳴く。
\item そうでないなら何もしない。
\end{itemize}
\end{itemize}
\end{exampleblock}
\end{frame}

\begin{frame}[fragile]{様々な条件}{}
\begin{exampleblock}{問題4}
スプライトを2つ追加します。
\begin{itemize}%\itemsep8pt
\item
「どこかの場所に移動する」を繰り返しながら、以下の2択を実行
\begin{itemize}%\itemsep8pt
\item 追加したスプライトとの距離が50以下ならニャーとなく。
\item そうでないなら何もしない。
\end{itemize}
\end{itemize}
\end{exampleblock}
\end{frame}

\begin{frame}[fragile]{}{}
もっといろいろな条件を使って動作を変更したい
\begin{itemize}%\itemsep8pt
\item 辺の数が2なら、描けないので、表示しない
\item 辺の数が20以上なら、区別がつかないので、表示しない
\item 奇数の辺と偶数番目の辺で、ペンの色を変えたい(2択)
\item 今日の日付が偶数なら、、、
\item 辺の長さを指定できるようにしたい
\end{itemize}
\end{frame}

\begin{frame}[fragile]{}{}
実行する度に変化する数値:変数
\begin{itemize}%\itemsep8pt
\item 「辺の数」変数
\item 「辺の数」変数
\item 「何番目の辺か」変数
\item 「今日の日付」変数
\item  「辺の長さ」変数
\end{itemize}
\end{frame}

\begin{frame}[fragile]{}{}
\begin{enumerate}%\itemsep8pt
\item 何番目の辺かを表す変数を作る
\item 何番目の辺かを表す変数を0に設定する
\item 辺を描いたら何番目の辺かを表す変数を1増やす
\item 何番目の辺かを表す変数が偶数か奇数かで2択をする
\end{enumerate}
\end{frame}

\end{document}
